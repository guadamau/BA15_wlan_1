% *****************************************************************************
% Nomenclature descriptions

\newcommand{\nonfreefirmwaredesc}{Bei Non-Free-Firmware handelt es sich um Gerätetreiber-Software, die nicht als komplett freie Software vertrieben wird. Das heisst, die Software ist proprietär und basiert auf herstellerbasierten Standards, die nicht veröffentlicht wurden.}

\newcommand{\gatewaydesc}{Netzwerkkomponente, die es ermöglicht, Computern in einem lokalen Netzwerk, den Zugriff in andere Netzwerke zu ermöglichen. Als bekanntestes Beispiel, kann an dieser Stelle die Verbindung von Computern in einem lokalen Netzwerk über einen Router als Gateway mit dem Internet erwähnt werden.}

\newcommand{\prpdesc}{\textbf{Parallel Redundancy Protocol}. Hochverfügbarkeitsnetzwerk, bei dem Netzwerkkomponenten über zwei voneinander unabhängige LANs kommunizieren. Beim Versand wird das Frame dupliziert und über beide LANs versandt. Das Duplikat wird vom Empfänger erkannt und verworfen.}
\newcommand{\dandesc}{\textbf{Double Attached Node oder Dual Attached Node}. Netzwerkteilnehmer eines Netzwerks, der über zwei Netzwerkinterfaces verfügt, die je direkt an einem LAN des Netzwerks angeschlossen sind.}
\newcommand{\danpdesc}{\textbf{Double Attached Node implementing PRP oder Dual Attached Node implementing PRP}. DAN in einem PRP-Netzwerk.}
\newcommand{\vdandesc}{\textbf{Virtual Double Attached Node oder Virtual Dual Attached Node}. Host, der über ein Netzwerkinterface an einer RedBox angeschlossen ist und somit darüber am PRP-Netzwerk teilnimmt. Für andere Netzwerkteilnehmer wird dieser Host wie ein DAN wahrgenommen.}
\newcommand{\sandesc}{\textbf{Single Attached Node}. Host, der nur an einem LAN des PRP-Netzwerks angeschlossen ist. Dieser kann mit allen DANs, VDANs und RedBoxen kommunizieren, jedoch nur mit anderen SANs, die am selben LAN angeschlossen sind. Ist zum Beispiel ein SAN nur am LAN A angeschlossen, kann dieser nur andere SANs erreichen, die auch am LAN A angeschlossen sind.}
\newcommand{\redboxdesc}{\textbf{Redundancy Box}. Ist mit beiden LANs des PRP-Netzwerks verbunden und bietet Anschlüsse für mehrere Hosts, damit diese über je 1 Netzwerkanschluss am PRP-Netzwerk teilnehmen können. Solche Hosts werden dann VDAN genannt.}
\newcommand{\rctdesc}{\textbf{Redundancy Control Trailer}. 4 Bytes langes Framefeld, um Frames, die über beide LANs eines PRP-Netzwerks verschickt werden, zu kennzeichnen.}
\newcommand{\lredesc}{\textbf{Link Redundancy Entitiy}. Einheit, die beide Netzwerkinterfaces eines DANs oder einer RedBox verbindet. Ist zuständig für die Frameduplikation und Duplikaterkennung.}
\newcommand{\hsrdesc}{\textbf{High Availability Seamless Redundancy}. Redundanzprotokoll für Ethernet basierte Netzwerke. HSR ist für redundant gekoppelte Ringtopologien ausgelegt. Die Datenübermittlung innerhalb eines HSR-Rings ist im Fehlerfall gewährleistet, wenn eine Netzwerkschnittstelle ausfallen sollte.}

\newcommand{\zerocopydesc}{Zero-copy beschreibt Computer-Operationen, bei denen die CPU nicht dafür zuständig ist, die Daten vom einen Speicher auf den anderen zu kopieren. Solche Operationen werden gebraucht, um beim Senden von Daten über ein Netzwerk an Prozessorleistung und Arbeitsspeicher-Gebrauch zu sparen.}

\newcommand{\memorymapdesc}{Einteilung in verschiedene Segmente. Die kleinste adressierbare Einheit eines Segments ist ein Byte.}
\newcommand{\syscalldesc}{Anfrage einer Applikation an den Kernel des Betriebssystems.}
\newcommand{\ecidesc}{\textbf{Ethernet Cable Interceptor}. Gerät für gezielte und reproduzierbare Störungen im Netzwerk, das bei einer Netzverbindung dazwischen geschaltet wird.}
\newcommand{\vszdesc}{\textbf{Virtual Memory Size}. Beschreibt den gesamten Speicher, der von einem Prozess benutzt wird, inklusive dem Speicher ausserhalb vom RAM und Shared-Libraries, die der Prozess verwendet.}
\newcommand{\rssdesc}{\textbf{Resident Set Size}. Beschreibt wie viel Speicher ein Prozess alloziert hat, der im Arbeitsspeicher ist. Zur RSS gehören zudem der Speicher, den Shared-Libraries des Prozesses im RAM belegen und den gesamten Stack- und Heap-Memory.}
\newcommand{\rxdesc}{\textbf{Reciever} / Empfänger.}
\newcommand{\txdesc}{\textbf{Transmitter} / Sender.}
\newcommand{\nicdesc}{\textbf{Network Interface Card} / Netzwerkkarte.}
\newcommand{\socketdesc}{Sockets bilden das Bindeglied zwischen Kommunikationsprotokollen und dem Betriebssystem. Ein Socket kann auch als Endpunkt einer Verbindung angesehen werden. Da in einem unixoiden System das Konzept gilt, ''Alles ist eine Datei.'', wird ein Socket seitens Betriebssystem als Datei repräsentiert. Der sendende Prozess schreibt die Daten in seine entsprechende Socket-Datei, wobei diametral dazu ein empfangender Prozess die Daten von seiner entsprechenden Socket-Datei liest.}
\newcommand{\headerdesc}{Eine Programm-Header-Datei hat die Dateiendung .h und enthält C-Funktionsdeklarationen sowie C-Makro-Definitionen. Die Header-Datei dient als Schnittstelle, zu den Quellcode-Dateien. Aufrufe von Funktionen, die in anderen Quellcodedateien implementiert wurden, werden durch das Einbinden der ensprechenden Header-Datei für den Compiler sichtbar.}
\newcommand{\cmacrodesc}{Ein Makro ist ein benanntes Code-Fragment und wird meist in einer Header-Datei definert. Der Name des Fragments kann beliebig in Quellcode-Dateien platziert werden. Der C-Präprozessor ersetzt vor dem Kompilieren die Fragment-Namen durch die Code-Fragmente.}
\newcommand{\payloaddesc}{Als Payload wird der reine Nutzdaten-Anteil eines Frames (Datalink-Layer) oder eines Paketes (Network-Layer, Transport-Layer, ...) definiert. Die Grösse des Payloads ist die Differenz der Gesamtpaketgrösse und der Grösse des entsprechenden Protokoll-Headers.}
\newcommand{\protoheaddesc}{Der Protokoll-Header enthält alle nötigen Metainformationen, die ein bestimmtes Kommunikationprotokoll braucht, um Daten austauschen zu können. Beispielfelder eines Protokoll-Headers: Quell- und Zieladresse, Port, Fragment-ID, etc.}
\newcommand{\cpointerdesc}{Ein Pointer in der Programmiersprache C, repräsentiert eine Speicheradresse.}
\newcommand{\fcsdesc}{Die \textbf{Frame Check Sequence} befindet sich am Ende eines Ethernet Frames. Sie hat eine Grösse von 4 Byte und wird von der sendenden Hardware (Ethernet-Controller) basierend auf dem Frame-Inhalt meist mittels CRC-Verfahrens berechnet und anschliessend am Ende eines jeden Frames platziert. Die empfangende Hardware berechnet für jedes empfangene Frame ebenfalls eine Frame Check Sequence und vergleicht diese mit der FCS am Ende des empfangenen Frames. Wenn beide FCS übereinstimmen, wird das Frame dem Netzwerk-Protokoll-Stack des Betriebssystems weitergereicht. Im Falle einer Divergenz der beiden FCS wird das Frame verworfen.}
\newcommand{\crcdesc}{\textbf{Cyclic Redundancy Check} ist ein Verfahren, basierend auf der modularen Polynomdivision (mod 2), welches dazu verwendet wird, einen Prüfwert über ein bestimmtes Datenfeld zu berechnen.}
\newcommand{\mmiodesc}{Ist ein Verfahren zur Kommunikation einer Zentraleinheit mit Peripheriegeräten. Die I/O-Register von elektronischen Bauelementen, mit denen angeschlossene Hardware gesteuert wird, werden in den Hauptspeicher-Adressraum abgebildet. Der Zugriff auf die Bauelemente kann dann über übliche Speicherzugriffsroutinen geschehen. Es werden keine besonderen Befehle benötigt wie bei der Realisierung der Ein-/Ausgabe mittels I/O-Ports am Prozessor. Gegenüber einem separaten I/O-Bus besitzt Memory Mapped I/O den Vorteil, dass man in der Regel über Strukturen und Pointer aus einer Hochsprache wie C oder C++ vollständig auf die Hardware zugreifen kann, ohne Teile des Programms in Assembler bzw. Maschinensprache schreiben zu müssen. \cite{MMIO_Page}}
\newcommand{\ethtdesc}{Software, die dazu verwendet werden kann, Gerätetreiber- und Hardware-Parameter von kabelgebundenen Netzwerk-Interfaces auszulesen, respektive zu modifizieren.}
\newcommand{\autonegdesc}{Ist ein Ethernet-Mechanismus, der es zwei miteinander verbundenen Geräten ermöglicht, automatisch die optimalen Grundeinstellungen bezüglich Geschwindigkeit, Duplex-Mode und Flow Control auszuhandeln. Während der Verhandlung teilen sich die beiden verbundenen Geräte ihre Eigenschaften mit und wählen dann die optimalen Einstellungen aus, welche beide Geräte unterstützen. Dieses Prozedere findet zu Beginn eines Verbindungsaufbaus zweier via Ethernetlink verbundenen Geräten statt.}
\newcommand{\mindesc}{Kurzform für Minimum}
\newcommand{\maxdesc}{Kurzform für Maximum}
\newcommand{\avgdesc}{Kurzform für Average, zu Deutsch Durchschnitt}
\newcommand{\userspacedesc}{Bereich im Arbeitsspeicher, wo normale Benutzerprozesse abgelegt respektive ausgeführt werden.}
\newcommand{\kernelspacedesc}{Bereich im Arbeitsspeicher, welcher ausschliesslich für Prozesse des Kernels reserviert ist. In diesem Bereich werden Kernelprozesse sowie privilegierte Systemprozesse abgelegt, respektive ausgeführt.}
\newcommand{\rttdesc}{\textbf{Round Trip Time} zu Deutsch Paketumlaufzeit. Damit ist die Zeit gemeint, welche benötigt wird, um von einer Quelle zum Ziel und wieder zurück zur Quelle zu gelangen.}

% *****************************************************************************
% Nomenclature entries

\nomenclature{Non-Free-Firmware}{\nonfreefirmwaredesc}
\nomenclature{Gateway}{\gatewaydesc}
\nomenclature{PRP}{\prpdesc}
\nomenclature{DAN}{\dandesc}
\nomenclature{DANP}{\danpdesc}
\nomenclature{VDAN}{\vdandesc}
\nomenclature{SAN}{\sandesc}
\nomenclature{RedBox}{\redboxdesc}
\nomenclature{RCT}{\rctdesc}
\nomenclature{LRE}{\lredesc}
\nomenclature{LRE}{\hsrdesc}
\nomenclature{Zero-copy Modus}{\zerocopydesc}
\nomenclature{Memory Map}{\memorymapdesc}
\nomenclature{System-Call}{\syscalldesc}
\nomenclature{ECI}{\ecidesc}
\nomenclature{VSZ}{\vszdesc}
\nomenclature{RSS}{\rssdesc}
\nomenclature{RX}{\rxdesc}
\nomenclature{TX}{\txdesc}
\nomenclature{NIC}{\nicdesc}
\nomenclature{Socket}{\socketdesc}
\nomenclature{Programm-Header-Datei}{\headerdesc}
\nomenclature{C-Makros}{\cmacrodesc}
\nomenclature{Payload}{\payloaddesc}
\nomenclature{Protokoll-Header}{\protoheaddesc}
\nomenclature{C-Pointer}{\cpointerdesc}
\nomenclature{FCS}{\fcsdesc}
\nomenclature{CRC}{\crcdesc}
\nomenclature{Memory Mapped I/O}{\mmiodesc}
\nomenclature{ethtool}{\ethtdesc}
\nomenclature{Autonegotiation}{\autonegdesc}
\nomenclature{Min}{\mindesc}
\nomenclature{Max}{\maxdesc}
\nomenclature{Avg}{\avgdesc}
\nomenclature{Userspace}{\userspacedesc}
\nomenclature{Kernelspace}{\kernelspacedesc}
\nomenclature{RTT}{\rttdesc}
